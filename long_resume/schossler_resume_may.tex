%-------------------------
% Resume in Latex
% Author : Jake Gutierrez
% Based off of: https://github.com/sb2nov/resume
% License : MIT
%------------------------
% - Applied conflict resolution strategies, when discussing misconceptions, regradings, etc., with students.


%I am a theoretical condensed matter physicist passionate about solving problems using statistics, quantum physics, and data science. Currently, I am fascinated by the fundamentals of machine and deep learning and their potential applications to improve the world. I believe the DLI is the perfect environment to flourish my understanding of the fundamentals of deep learning and to motivate me to endeavor in this field.


% I am a theoretical condensed matter physicist passionate about solving problems using statistics, quantum physics, and data science. Currently, I am fascinated by the fundamentals of machine and deep learning and their potential applications. I believe the DLI is an environment to flourish my understanding of the fundamentals of deep learning and to motivate me in this field.

\documentclass[letterpaper,11pt]{article}

\usepackage{latexsym}
\usepackage[empty]{fullpage}
\usepackage{titlesec}
\usepackage{marvosym}
\usepackage[usenames,dvipsnames]{color}
\usepackage{verbatim}
\usepackage{enumitem}

\usepackage[svgnames]{xcolor}
\usepackage[hidelinks, colorlinks = true, urlcolor = DarkSlateBlue,]{hyperref}
% %color hyperlinks
% \usepackage[colorlinks = true,
%             linkcolor = blue,
%             urlcolor  = blue,
%             citecolor = blue,
%             anchorcolor = blue]{hyperref}
\usepackage{fancyhdr}
\usepackage[english]{babel}
\usepackage{tabularx}
% \usepackage{fontawesome5}
\usepackage{multicol}
\setlength{\multicolsep}{-3.0pt}
\setlength{\columnsep}{-1pt}
\input{glyphtounicode}
\usepackage{outlines}
\usepackage{tikz}

% \usepackage{academicons}
\usepackage{fontawesome5}

\newcommand*{\googlescholarsocialsymbol}{{\small\faGraduationCap}~}
%----------FONT OPTIONS----------
% sans-serif
% \usepackage[sfdefault]{FiraSans}
% \usepackage[sfdefault]{roboto}
% \usepackage[sfdefault]{noto-sans}
% \usepackage[default]{sourcesanspro}

% serif
% \usepackage{CormorantGaramond}
% \usepackage{charter}


\pagestyle{fancy}
\fancyhf{} % clear all header and footer fields
\fancyfoot{}
\renewcommand{\headrulewidth}{0pt}
\renewcommand{\footrulewidth}{0pt}

% Adjust margins
\addtolength{\oddsidemargin}{-0.6in}
\addtolength{\evensidemargin}{-0.5in}
\addtolength{\textwidth}{1.19in}
\addtolength{\topmargin}{-.7in}
\addtolength{\textheight}{1.4in}

\urlstyle{same}

\raggedbottom
\raggedright
\setlength{\tabcolsep}{0in}

\newcommand{\vspaceDefaultAbove}{\vspace{-2pt}}
\newcommand{\vspaceDefaultBelow}{\vspace{-7pt}}
\newcommand{\vspaceDefaultSpace}{\vspace{4pt}}

% Sections formatting
% \titleformat{\section}{
%   \vspace{-8pt}\scshape\raggedright\large\bfseries
% }{}{0em}{}[\color{black}\titlerule \vspace{-5pt}]
\titleformat{\section}{
  \vspace{-9pt}\raggedright\large\bfseries
}{}{0em}{}[\color{black}\titlerule \vspace{-5pt}]

% Ensure that generate pdf is machine readable/ATS parsable
\pdfgentounicode=1
\usepackage{ragged2e}


\usepackage{setspace}
\setstretch{0.95}
%-------------------------
% Custom commands
\newcommand{\resumeItem}[1]{\justifying
  \item{{\begin{spacing}{0.96}
    {#1\vspaceDefaultAbove}\end{spacing}\vspace{0pt}
  }}
}

\newcommand{\SubItem}[1]{
   {\setlength\itemindent{6pt} \item[- \hspace{-6pt}] { #1}}
}

\newcommand{\resumeSubItem}[1]{
  \vspaceDefaultAbove\SubItem{
    {#1 \vspaceDefaultAbove}
  }
}

\newcommand{\classesList}[4]{
    \item{{
        {#1 #2 #3 #4 \vspaceDefaultAbove}
  }}
}

\newcommand{\resumeSubheading}[4]{
  \vspaceDefaultAbove\item
    \begin{tabular*}{1.0\textwidth}[t]{l@{\extracolsep{\fill}}r}
      \textbf{#1\vspace{-1pt}} & \textbf{{#2\vspace{-1pt}}} \\
      {{#3}} & {{#4}} \\
    \end{tabular*}\vspace{-6pt}
}


\newcommand{\resumeSubSubheading}[2]{
    \item
    \begin{tabular*}{0.97\textwidth}{l@{\extracolsep{\fill}}r}
      \textit{{#1} } & \textit{{#2}} \\
    \end{tabular*}\vspaceDefaultBelow
}

%________________________EVENT_______________________________________
\newcommand{\resumeEvent}[2]{
      \vspaceDefaultAbove\begin{tabular}{ >{\raggedright\let\newline\\\arraybackslash\hspace{0pt}}p{6.9in} wr{0.8in}}
      {\raise .5ex\hbox{\tiny$\bullet\ \ $}}{#1}  & \textbf{{#2}}\\
     \end{tabular}\vspace{0pt}
}

\newcommand{\resumeEventListStart}{\begin{itemize}[leftmargin=0.2in]}
\newcommand{\resumeEventListEnd}{\end{itemize}}
%________________________EVENT_______________________________________



\newcommand{\resumeProjectHeading}[2]{
      \vspace{-6pt}\item
    \begin{tabular*}{1.001\textwidth}{l@{\extracolsep{\fill}}r}
      {#1} & \textbf{{#2}} \\
    \end{tabular*}\vspaceDefaultBelow
}


\newcommand{\resumeSubProjectHeading}[2]{
    \item
    \begin{tabular*}{0.987\textwidth}{l@{\extracolsep{\fill}}r}
       #1 & {#2}\\
    \end{tabular*}\vspaceDefaultBelow
}


\renewcommand\labelitemi{$\vcenter{\hbox{\tiny$\bullet$}}$}
\renewcommand\labelitemii{$\vcenter{\hbox{\tiny$\bullet$}}$}

\newcommand{\resumeSubHeadingListStart}{\begin{itemize}[leftmargin=0.0in, label={}]}
\newcommand{\resumeSubHeadingListEnd}{\end{itemize} \vspace{1pt}}

\newcommand{\resumeSubSubHeadingListStart}{\begin{itemize}[leftmargin=0.2in,label={}]}
\newcommand{\resumeSubSubHeadingListEnd}{\end{itemize}}

\newcommand{\resumeItemListStart}{\begin{itemize}[leftmargin=0.125in,rightmargin=0in]}
\newcommand{\resumeItemListEnd}{\end{itemize}\vspaceDefaultBelow\vspaceDefaultAbove}

\newcommand{\vspaceSectionAbove}{\vspaceDefaultAbove}
\newcommand{\vspaceSectionBelow}{\vspaceDefaultBelow}

%color hyperlinks


%-------------------------------------------
%%%%%%  RESUME STARTS HERE  %%%%%%%%%%%%%%%%%%%%%%%%%%%%


\begin{document}
%----------HEADING----------
% % % \begin{tabular*}{\textwidth}{l@{\extracolsep{\fill}}r}
% % %   \textbf{\href{http://sourabhbajaj.com/}{\Large Sourabh Bajaj}} & Email : \href{mailto:sourabh@sourabhbajaj.com}{sourabh@sourabhbajaj.com}\\
% % %   \href{http://sourabhbajaj.com/}{http://www.sourabhbajaj.com} & Mobile : +1-123-456-7890 \\
% % % \end{tabular*}
% % % \aiGoogleScholar
% \begin{center}
% 	\vspace{-5pt} 
%     \begin{tabular*}{\textwidth}{ccc}
%      ~  & {{\huge \scshape Matheus Schossler}} ~ ~ &
%       \begin{tabular*}{\textwidth}{@{}ll@{}ll@{}}
%                \href{mailto:mschossler@wustl.edu}{\raisebox{-0.2\height}\faEnvelope\  {mschossler@wustl.edu}} ~ &  \href{https://www.linkedin.com/in/mschossler}{\raisebox{-0.2\height}\faLinkedin\ {linkedin.com/in/mschossler}} ~  & \href{https://scholar.google.com/citations?user=MVXXEdIAAAAJ}{\raisebox{-0.2\height}\googlescholarsocialsymbol {Matheus Schossler}}  \\
%                \href{tel:+13144750151}{ {\small \faPhone\ (314) 475-0151} } & \href{https://github.com/mschossler}{\raisebox{-0.2\height}\faGithub\ {github.com/mschossler}} &  \href{https://physics.wustl.edu/people/matheus-schossler}{\raisebox{-0.2\height}\googlescholarsocialsymbol {wustl.edu}}
%       \end{tabular*} \\
%     \end{tabular*}
% \end{center}
% %%%%
\begin{center}
	\vspace{-5pt} 
    \begin{tabular*}{\textwidth}{ccc}
     ~ ~ ~ ~ ~ ~ & {{\huge \scshape Matheus Schossler}} ~ ~ ~ &
      \begin{tabular*}{\textwidth}{@{}ll@{}ll@{}}
               \href{mailto:mschossler@wustl.edu}{\raisebox{-0.1\height}\faEnvelope\  {mschossler@wustl.edu}} ~ &  \href{https://www.linkedin.com/in/mschossler}{\raisebox{-0.008\height}\faLinkedin\ {linkedin.com/in/mschossler}} ~ \\
               \href{tel:+13144750151}{ {\small \faPhone\ (314) 475-0151} } & \href{https://github.com/mschossler}{\raisebox{-0.008\height}\faGithub\ {github.com/mschossler}} 
      \end{tabular*} \\
    \end{tabular*}
\end{center}
%
%
%---------------------------------------EDUCATION-----------
\section{\uppercase{Education}}
%
\resumeSubHeadingListStart
\resumeSubheading
{Washington University in St. Louis}{Aug. 2017 -- Present}
{PhD Physics Candidate (May 2023) \& MA Physics (Aug. 2019), GPA: 3.96}{Saint Louis, MO, USA}


\vspace{-0.1cm}
Thesis Supervisor: Dr. Alexander Seidel
\vspace{-0.3cm}
% \resumeSubheading
% {Washington University in St. Louis}{Aug. 2017 -- Aug. 2019}
% {, GPA: 3.96}{Saint Louis, MO, USA}
\resumeItemListStart
\resumeItem{Relevant Coursework: Adv. statistical physics II}
\resumeItem{Audited classes in ML: Machine learning, Andrew NG,  Stanford {\small CS229}; Statistical learning theory \& applications, {\small MIT 9.520/6.860}; Fundamentals of deep learning, {\small NVIDIA deep learning inst.}, (TensorFlow); Deep learning specialization, Andrew NG, {\small Coursera}.}
\resumeItem{\emph{Extracurricular:} Machine Learning Club; Diversity, Equity and Inclusion Committee; Physics Department Mentor.}
\resumeItemListEnd
\vspaceDefaultSpace
\resumeSubheading
{University of S{a}o Paulo}{Feb. 2015 -- Jul. 2017}
{MS Physics, GPA: A}{S{a}o Carlos, SP, Brazil}
\resumeItemListStart
\resumeItem{Thesis: exact and numerical calculations of quantum spin correlations.[\href{https://www.teses.usp.br/teses/disponiveis/76/76131/tde-15092017-090718/en.php}{link}]\vspaceDefaultAbove}
\resumeItem{Relevant Coursework: Adv. statistical physics I; Adv. Quantum Mechanics.}% 		\item{ Adv. statistical mechanics I, II}
% 		\item{ Adv. quantum mechanics}
% 		\item{ Adv. mathematical physics I}
% 		\item{ Quantum many-particle systems}
% 		\item{ Quantum field theory I, II}
% 		\item{ Solid state physics}}
\resumeItemListEnd
\vspaceDefaultSpace
\resumeSubheading
{University of S{a}o Paulo}{Mar. 2011 -- Dec. 2014}
{BS Physics, GPA: 8.3/10}{S{a}o Carlos, SP, Brazil}%\vspaceDefaultBelow is implicit
\resumeItemListStart
\resumeItem{Relevant Coursework: Intro to computer programming (C), Numerical analysis (C), Intro to computational physics (Fortran), Wolfram programming language, Statistical physics, Linear algebra, and several other math courses, Intro to electronics.}  % 		\item{ }
% 		\item{ Numerical analysis}
% 		\item{ Intro to computational physics}
% 		\item{ Wolfram programming language}
% 		\item{ Intro to electronics}
% 		\item{ Statistical physics}
% 		\item{ Intro to solid state physics}
% 		\item{ Quantum mechanics I, II}
% 		%\item{ Calculus I, II, III}
% 		%\item{ Analytic geometry}
% 		\item{ Linear algebra}}
% \vspace{1pt}
\resumeItemListEnd
% \vspaceDefaultSpace
\resumeSubHeadingListEnd
%
%--------------------------------------EXPERIENCE-----------
\section{\uppercase{Experience}}
\resumeSubHeadingListStart
\resumeSubheading
{Washington University in St. Louis}{Aug. 2017 -- Present}{PhD Researcher and Teaching Assistant}{Saint Louis, MO, USA}
%{PhD Researcher in Statistical, Quantum and Computational Physics; and Teaching Assistant}{Saint Louis, MO, USA}
\resumeItemListStart
\resumeItem{{Developed an exact algorithm to model various topological quantum states with a $\infty$-linked tensor network.[\href{https://arxiv.org/abs/2111.09988}{link}]}}
%
\resumeItem{Constructed new self-consistent numerical models in \emph{Python} for statistical inference of experimental data.}
% \resumeItem{Constructed 4 new self-consistent numerical models in \emph{Python} for statistical inference of experimental data.}
%
\resumeItem{Predicted energy levels and probability distributions of quantum states using \emph{parallel computing} with 30x faster calculations.}
%
\resumeItem{Conducted extensive data analysis of simulation and real data from experiments at the frontier of  quantum physics knowledge.}
%
% \resumeItem{Analyzed scalability of predictions to reliably extract electric field dumping due interactions.}
%
\resumeItem{Designed graphics and schematic diagrams for data visualization using \emph{matplotlib}, and \emph{Mathematica}.}
%
\resumeItem{Presented results at conferences to 100+ attendees.}
%                   
% \resumeItem{Held teaching assistant positions for several courses: Statistical mechanics (graduate level); Electromagnetism II; Adv. mathematical physics I \& II (graduate level); Introductory physics, etc.}
\resumeItem{Teaching assistant for several courses: Statistical mechanics (graduate level); Electromagnetism II; Adv. mathematical physics I \& II (graduate level); Introductory physics, etc.}
\resumeItemListEnd
\begin{comment}
\resumeItemListStart
%
% \resumeItem{{Fractional quantum Hall states modeling [\href{https://arxiv.org/abs/2111.09988}{link}]} $|$\emph{Discrete mathematics, tensor network}} %$|$Aug. 2017 -- Present}
\resumeItem{{Developed an exact algorithm to model various topological quantum states with a $\infty$-linked tensor network.[\href{https://arxiv.org/abs/2111.09988}{link}]}} %$|$Aug. 2017 -- Present}
% \resumeItem{{Developed an exact algorithm to model various topological quantum states as a tensor network with nodes connected by an infinite dimension link using \emph{Mathematica}, \emph{physics}, \emph{statistics}, \emph{math}, and \emph{tensor network}.[\href{https://arxiv.org/abs/2111.09988}{link}]}} %$|$Aug. 2017 -- Present}
%
% \resumeItem{{Cyclotron resonance in dual-gated bilayer graphene (in progress)} $|$\emph{Python, Mathematica, Matlab}}% $|$May 2020 -- Present}
% \resumeItem{Predicted energy levels and probability distributions for states in bilayer graphene using \emph{pandas}, and \emph{parallel computing} in \emph{clusters}.}
\resumeItem{Predicted energy levels and probability distributions of quantum states using \emph{parallel computing} with 30x faster calculations.}
\resumeItem{Conducted extensive data analysis of simulation and real data from experiments.}
\resumeItem{Designed graphics and schematic diagrams for data visualization using \emph{matplotlib}, and \emph{Mathematica}.}
% \resumeItem{Predicted energy levels and probability distributions for states of bilayer graphene under strong magnetic field utilizing python libraries such as \emph{pandas} and \emph{multiprocessing} for faster numerical calculations in computer clusters.}\vspaceDefaultAbove
%
\resumeItem{Constructed 4 new self-consistent numerical models using \emph{Python} for statistical inference of experimental data.}
\resumeItem{Analyzed scalability of predictions to reliably extract electric field dumping due interactions.}% ADD TO GITHUB.
% STORY: I initially found the scalability working numerically because as I increased the       number of LL taken into account (L), the distance between curves U vs U_0 with different L decreased as a function of L.
\resumeItem{Presented results at conferences to 100+ attendees.}
%                   
\resumeItem{Teaching assistant for several courses: Statistical mechanics (graduate level); Electromagnetism II; Adv. mathematical physics I \& II (graduate level); Introductory physics, etc.}
% STORY: % Prepared and lectured classes, and graded assignments. 
         % Communicated with students to resolve course conflicts.
         % Applied conflict resolution strategies, when discussing misconceptions, regradings, etc., with students.
\resumeItemListEnd
\end{comment}
%
%
\vspaceDefaultSpace
\resumeSubheading
{University of S{a}o Paulo}{Feb. 2015 -- Jul. 2017}{MSc Researcher and Teaching Assistant}{S{a}o Carlos, SP, Brazil}
% {MSc Researcher in Statistical, Quantum and Computational Physics; and Teaching Assistant}{S{a}o Carlos, SP, Brazil}
\resumeItemListStart
% \resumeItem{Thesis: exact and numerical calculations of quantum spin correlations.[\href{https://www.teses.usp.br/teses/disponiveis/76/76131/tde-15092017-090718/en.php}{link}].}% \href{https://bv.fapesp.br/en/bolsas/160881/time-dependent-correlations-of-tensor-operators-in-quantum-spin-chains/}{Scholarship from the S{a}o Paulo Research Foundation}.\vspaceDefaultAbove}
\resumeItem{Developed and applied exact and numerical algorithms to calculate statistical correlations in large quantum systems.} %using \emph{linear response theory}, \emph{Feynman diagrams}, and \emph{Mathematica}.}
%
\resumeItem{Implemented gradient descent to predict asymptotic behaviors of correlations.}
\resumeItem{Tested various numerical models to predict correlations with highly oscillating kernels.}
\resumeItem{Teaching assistant for Electromagnetism I and Introductory physics for engineers.}
% \resumeItem{Developed and applied exact and approximate algorithms to calculate correlations for large systems. $|$ {linear response theory, Feynman diagrams, Mathematica}.}
%
% \resumeItem{Implemented gradient descent to predict asymptotic behaviors of correlations, including highly oscillating kernels. $|$ {Mathematica} }
%
% \resumeItem{M.Sc. thesis in LaTeX with various new plots and diagrams, and utilized Bash scripts to organize files.[\href{https://www.teses.usp.br/teses/disponiveis/76/76131/tde-15092017-090718/pt-br.php}{link}]}
%
% \resumeItem{Teaching responsibilities as above. Courses: Physics II; Electromagnetism I.}
\resumeItemListEnd



%
% %
\vspaceDefaultSpace
\resumeSubheading
{University of S{a}o Paulo}{Jan. 2014 -- Dec. 2014}
{Undergraduate Researcher}{S{a}o Carlos, SP, Brazil}
\resumeItemListStart
\resumeItem{Developed and applied exact algorithms to calculate discrete Fourier transform of correlation functions of a large spin system in the frequency domain.  \href{https://bv.fapesp.br/en/bolsas/147280/dynamics-of-two-body-operators-in-exactly-solvable-spin-chains/}{Scholarship from the S{a}o Paulo Research Foundation}.}
\resumeItemListEnd
%
% \vspaceDefaultSpace
% % % \resumeSubheading
% % % {University of S{a}o Paulo}{Mar. 2013 -- Dec. 2013}
% % % {Computer lab supervisor intern - S{a}o Carlos institute of physics}{S{a}o Carlos, SP, Brazil}
% % % \resumeItemListStart
% % % \resumeItem{Collected usage data as part of a team to optimize hours of operation to match student schedules.}
% % % %
% % % \resumeItem{Provided tech support, with duties including: assurance of computer security, oversight of software installation and maintenance, and equipment inventory.}
% % % % \resumeItem{Assisted in the installation, configuration, and maintenance of software, printers, scanners, and peripherals. Duties included ensuring the security of computers, care of the lab equipment, upkeep of supplies, and providing a smooth operation.}
% % % \resumeItemListEnd
\resumeSubHeadingListEnd
% \vspaceSectionBelow
%
%--------------------------------------SKILLS-----------
\section{\uppercase{Skills}}
\textbf{Modeling:}{ statistics, stochastic calculus,  data analysis, machine learning, deep learning, NumPy, pandas, PyTorch, TensorFlow, matplotlib, scikit-learn, SciPy, Mathematical modeling,  parallel computing, time series}

% \textbf{Audited classes in ML:}{ Machine learning, Andrew NG,  Stanford {\small CS229}; Statistical learning theory \& applications, {\small MIT 9.520/6.860}; Fundamentals of deep learning, {\small NVIDIA deep learning inst.}; Deep learning specialization, Andrew NG, {\small Coursera}.}

\textbf{Programming languages$\,$({\small YOE}):}{ Python$\,${\small(5)}, C$\,${\small(1)}, SQL$\,${\small(0.2)}, Matlab$\,${\small(0.5)}, Fortran$\,${\small(1)}, Wolfram$\,$Mathematica$\,${\small(6)} }%, Bash$\,$(1)
%  Python, C, Fortran, Bash, SQL, Matlab, Wolfram Mathematica

\textbf{Technologies/Frameworks:}{ Linux, Jupyter notebooks, PyCharm, Google Colab,  LaTeX, Gnuplot}

\textbf{Soft skills:}{ Problem solving, research, detail-focused, self learning, teamwork, communication, teaching}%, Portuguese,  basic Spanish

%
\vspace{1.5pt}
%--------------------------------PROJECTS-----------
\section{\uppercase{Additional Projects}}
\resumeSubHeadingListStart
%     
\resumeProjectHeading
{\textbf{Machine Learning projects} $|$ {PyTorch, TensorFlow, scikit-learn, SciPy, Pandas, Numpy} }{Feb. 2018 -- Present}
\resumeItemListStart
\resumeItem{Predicted among 15,000 customers the most likely ones to churn based on their transaction activity history over a two years period by employing K-means algorithm to cluster businesses according to their attributes. [\href{https://github.com/mschossler/TimeSeriesClustering}{GitHub}]}

\resumeItem{Studied computer vision, natural language processing, and other ML and DL algorithms in small real-world projects using PyTorch, TensorFlow, scikit-learn, and SciPy.}
\resumeItemListEnd
%%
\vspaceDefaultAbove\vspaceDefaultAbove
%
\resumeProjectHeading
% {\textbf{Process data from Robinhood to calculate income taxes} $|$ \emph{Python, pandas}}{March 2020}
{\textbf{Robinhood account activity data to Glacier Tax application [\href{https://github.com/mschossler/robinhood-account-activity-data-converter-to-glacier-tax}{GitHub}]} $|$ {Python, pandas}}{March 2020}
\resumeItemListStart
% \resumeItem{Created an application in python for cleaning and processing account activity data from Robinhood to be uploaded to Glacier Tax.}% The application finds and matches stocks that were sold in 2020 with stocks previously bought to calculate the profit and ultimately the income taxes for stock transactions for nonresident aliens. It also exports the data in the format required to be uploaded into glaciertax.com.
\resumeItem{Created an app in python for cleaning and processing account activity data from Robinhood to be uploaded to Glacier Tax. The application finds and matches stocks that were sold in 2020 with stocks previously bought to calculate the profit and ultimately income taxes for stocks for nonresident aliens. It also exports the data in the format required to be uploaded into glaciertax.com. I recently started building an API using Flask and Python for better automation.}% It also exports the data in the format required to be uploaded into glaciertax.com.}
\resumeItemListEnd
%
\vspaceDefaultAbove\vspaceDefaultAbove
%
\resumeProjectHeading
{\textbf{Fluctuation relations [\href{https://github.com/mschossler/fluctuation-relations}{GitHub}]} $|$ {statistics, physics} }{Nov. 2018 -- Dec. 2018}
\resumeItemListStart
% \resumeItem{Derived the Jarzynski equality. This fluctuation relation directly connects equilibrium states with non-equilibrium states for various systems. This relation, and the Crooks fluctuation theorem, were obtained by properly accounting the fluctuations (noise) for each sample of the ensemble of non-equilibrium trajectories space. Connections with machine learning algorithms are pointed out.}
\resumeItem{Derived the Jarzynski equality fluctuation relation. It connects equilibrium states with non-equilibrium states for various systems by properly accounting the noise for each sample of the ensemble of non-equilibrium trajectories space. Links with machine learning algorithms are pointed out.}
\resumeItemListEnd
%
% \pagebreak
% \vspaceDefaultAbove\vspaceDefaultAbove
%
% \vspace{0pt}
\resumeSubHeadingListEnd
% \pagebreak
\begin{comment}
%-----------------------------------INVOLVEMENT---------------
\section{\uppercase{Extracurriculars}}
\resumeSubHeadingListStart
%
% \resumeSubheading{Audited classes in ML}{Aug 2018 -- Present}{}{}
% \vspace{-5pt}
% \resumeItemListStart
% \begin{multicols}{2}
% \resumeItem{Stanford CS229: machine learning, Andrew NG\vspaceDefaultAbove}
% \resumeItem{NVIDIA DLI: fundamentals of deep learning}
% \resumeItem{Coursera: deep learning specialization, Andrew NG\vspaceDefaultAbove}
% \resumeItem{MIT 9.520/6.860: statistical learning theory and applications}
% \end{multicols}
% \resumeItemListEnd
% \vspaceDefaultSpace
% \resumeSubHeadingListEnd
% \pagebreak
% \resumeSubHeadingListStart
\resumeSubheading{Machine Learning Club}{Jan. 2018 -- Mar. 2020}{Member}{Washington University in St. Louis}
\resumeItemListStart
\resumeItem{Studied machine learning techniques in teams with other researchers; attended lectures and events about machine learning and its applications. Sponsored by the Physics department.}
%STORY: Check machine learning bootcamp folder
\resumeItemListEnd

%
\vspaceDefaultSpace
\resumeSubheading{Diversity, Equity and Inclusion Committee}{Aug. 2021 -- Present}{Member}{Dept. Physics, Washington University in St. Louis}
\resumeItemListStart
\resumeItem{Advocated with the university leadership for the maintenance and expansion of an academic program (JPP) promoting minority students in STEM.}
\resumeItemListEnd
%
\vspaceDefaultSpace
\resumeSubheading{Physics Department Mentors}{May 2020 -- Present}{Member}{Washington University in St. Louis}
\resumeItemListStart
\resumeItem{Helped new students transitioning into graduate school and provided support throughout their first few years by hosting several orientation events as well as individual advising.}
\resumeItemListEnd
%
\resumeSubHeadingListEnd
% \vspaceDefaultBelow
\end{comment}
%--------------------------------------------TALKS AND EVENTS-----------
\section{\uppercase{Talks and Events}}%Comunication Skills
%     \resumeSubHeadingListStart
\resumeEvent{\textbf{Duke Machine Learning Winter School: Computer Vision (MLWS-CV)} $|$ \emph{PyTorch, CNN's}}{Jan. 2022}
%
\resumeEvent{\textbf{Conference on Neural Information Processing Systems (NeurIPS 2021)}}{Dec. 2021}
%
\resumeEvent{March Meeting 2021, APS, online, \emph{Fractional quantum Hall states: frustration-free parent Hamiltonians and infinite-bond-dimension matrix-product-states}}{Mar. 2021}
%
\resumeEvent{March Meeting 2020, APS, online, \emph{From CFT matrix product states to parent Hamiltonians}}{  Mar. 2020}
%      
\resumeEvent{National high magnetic field laboratory winter school, Tallahassee FL, \emph{From CFT matrix product states to FQHS parent Hamiltonians}}{Jan. 2020}
%      
\resumeEvent{March Meeting 2019, APS, \emph{A recursion approach to thin cylinder approximants for fractional quantum Hall states}}{Mar. 2019}
%
\resumeEvent{Workshop on Quantum Non-Equilibrium Phenomena, IIP-UFRN, \emph{Dynamical correlation functions of a two-spin operator in quantum spin chains}}{Jun. 2016}
% \vspaceSectionBelow
%
%-----------------------------------PUBLICATIONS-----------
\section{\uppercase{Publications}}
\resumeItemListStart
\resumeItem{ M. Schossler, \emph{et al}. \emph{Theoretical description of the Cyclotron Resonance in Dual-Gated Bilayer Graphene}.[\href{https://meetings.aps.org/Meeting/MAR22/Session/T00.128}{in progress}]}\vspaceDefaultBelow
%
\resumeItem{ M. Schossler, \emph{et al}. \emph{The inner workings of fractional quantum Hall parent Hamiltonians: An MPS point of view}. 2021.[\href{https://arxiv.org/abs/2111.09988}{link}]}\vspaceDefaultBelow
%
%\resumeItem{ B. Russell, M. Schossler, J. Balgley, T. Taniguchi, K. Watanabe, A. Seidel, Y. Barlas, E. Henriksen. Spin-to-valley-polarized transition in bilayer graphene. In progress.}
%
\vspaceDefaultAbove
\resumeItem{ M. Schossler, M.Sc. thesis. University of S{a}o Paulo, S{a}o Carlos, 2017. \emph{Dynamics of two-spin operators in the XX model.}[\href{https://www.teses.usp.br/teses/disponiveis/76/76131/tde-15092017-090718/en.php}{link}]} \vspaceDefaultBelow%  Advisor:  Rodrigo Goncalves Pereira.
%
\vspaceDefaultAbove
% \resumeItem{ Six \href{https://scholar.google.com/citations?user=MVXXEdIAAAAJ&hl=en&oi=sra}{conference proceedings}. APS March Meeting 2019, 2020 \& 2021. Physics Week, USP, Sao Carlos, 2014 \& 2015.} 
\resumeItem{ Six conference proceedings. APS March Meeting 2019, 2020 \& 2021. Physics Week, USP, Sao Carlos, 2014 \& 2015.[\href{https://scholar.google.com/citations?user=MVXXEdIAAAAJ&hl=en&oi=sra}{scholar}]} \vspaceDefaultBelow%  Advisor:  Rodrigo Goncalves Pereira.
\resumeItemListEnd
\vspace{2pt}
% \vspaceDefaultBelow
%
%
%--------------------------------------HONORS-----------------
\section{\uppercase{Honors \& Awards}}
\resumeEvent{Award \emph{IFT-ICTP-SAIFR Young Physicist} (South American competition), 4th place, IFT-UNESP, S{a}o Paulo.}{Mar. 2016}
%
% \resumeEvent{Prize \emph{Yvonne Primerano Mascarenhas} for best presentation, 3rd place, graduate physics week, IFSC, University of S{a}o Paulo.}{Oct. 2015}
%
\resumeEvent{\href{https://bv.fapesp.br/en/bolsas/160881/time-dependent-correlations-of-tensor-operators-in-quantum-spin-chains/}{Scholarship 15/05644-9} - Masters grant. S{a}o Paulo Research Foundation (FAPESP). \emph{Time-dependent correlations of tensor operators in quantum spin chains.}}{Sept. 2015}
%
\resumeEvent{Prize \emph{Yvonne Primerano Mascarenhas} for best presentation, 2nd place, undergraduate physics week, IFSC, University of S{a}o Paulo.}{Oct. 2014}
%
\resumeEvent{\href{https://bv.fapesp.br/en/bolsas/147280/dynamics-of-two-body-operators-in-exactly-solvable-spin-chains/}{Scholarship 13/21168-7} - Scientific Initiation grant. S{a}o Paulo Research Foundation (FAPESP). \emph{Dynamics of two-body operators in exactly solvable spin chains}.}{Jan. 2014}
%
\resumeEvent{\emph{Bronze Medal}, Brazilian Olympics of astronomy and astronautics (OBA!), Ministry of Education, Brazil.}{May. 2010}
% \vspaceSectionBelow
% \vspaceSectionBelow
%
%-----------------------------------VOLUNTEERING-----------
\section{\uppercase{Volunteering}}
BJC HealthCare, as \textbf{Data Manager}; St. Louis Inter-Faith Committee on Latin America {\small (IFCLA)} \& Prof. Sebasti{a}o de Oliveira Rocha, public state high school, as \textbf{English/Portuguese translator}.
% \section{\uppercase{Volunteering}}
% % BJC HealthCare, as \textbf{Data Manager}; St. Louis Inter-Faith Committee on Latin America {\small (IFCLA)} \& Prof. Sebasti{a}o de Oliveira Rocha, public state high school, as \textbf{English/Portuguese translator}.
% % \textbf{BJC HealthCare}, Data Manager; \textbf{St. Louis Inter-Faith Committee on Latin America {\small (IFCLA)}} \& \textbf{Prof. Sebasti{a}o de Oliveira Rocha, public state high school}, English/Portuguese translator.
% \resumeSubHeadingListStart
% \resumeProjectHeading{\textbf{BJC HealthCare}, Data Manager}{Feb. 2021}
% % 
% % \vspace{2pt}
% % \hspace{9pt}Data Manager\vspaceDefaultBelow
% %{Data Manager}{Saint Louis, MO, USA}
% % \resumeItemListStart
% % \resumeItem{Worked as data manager in a vaccination clinic from BJC HealthCare during the COVID-19 pandemic.}
% % \resumeItemListEnd
% %
% \vspace{-18pt}
% \resumeProjectHeading{\textbf{St. Louis Inter-Faith Committee on Latin America {\small (IFCLA)}}, English/Portuguese translator}{Aug. 2020}
% % 
% % \vspace{2pt}
% % \hspace{9pt}English/Portuguese translator\vspaceDefaultBelow
% %{English/Portuguese translator}{Saint Louis, MO, USA}
% % \resumeItemListStart
% % \resumeItem{Intermediated the distribution of \$500 in aid during the COVID-19 pandemic.}
% % \resumeItemListEnd
% %
% \vspace{-18pt}
% \resumeProjectHeading{\textbf{Prof. Sebasti{a}o de Oliveira Rocha, public state high school}, English/Portuguese translator}{May 2016 -- June 2016}
% % 
% % \vspace{2pt}
% % \hspace{9pt}English/Portuguese translator\vspaceDefaultBelow
% %{English/Portuguese translator}{S{a}o Carlos, SP, Brazil}
% % \resumeItemListStart
% % \resumeItem{Translated lectures (English/Portuguese) from international exchange students regarding global warming for public high schoolers.}
% % \resumeItemListEnd
% \resumeSubHeadingListEnd
% % % \section{\uppercase{Volunteering}}
% % % \resumeSubHeadingListStart
% % % \resumeSubheading
% % % {BJC HealthCare}{Feb. 2021}
% % % {Data Manager}{Saint Louis, MO, USA}
% % % \resumeItemListStart
% % % \resumeItem{Worked as data manager in a vaccination clinic from BJC HealthCare during the COVID-19 pandemic.}
% % % \resumeItemListEnd
% % % %
% % % \vspaceDefaultSpace
% % % \resumeSubheading
% % % {St. Louis Inter-Faith Committee on Latin America (IFCLA)}{Aug. 2020}
% % % {English/Portuguese translator}{Saint Louis, MO, USA}
% % % \resumeItemListStart
% % % \resumeItem{Intermediated the distribution of \$500 in aid during the COVID-19 pandemic.}
% % % \resumeItemListEnd
% % % %
% % % \vspaceDefaultSpace
% % % \resumeSubheading
% % % {Professor Sebasti{a}o de Oliveira Rocha, public state high school}{May 2016 -- June 2016}
% % % {English/Portuguese translator}{S{a}o Carlos, SP, Brazil}
% % % \resumeItemListStart
% % % \resumeItem{Translated lectures from international exchange students regarding global warming for public high schoolers.}
% % % \resumeItemListEnd
% % % \resumeSubHeadingListEnd
%---------------------------------RELEVANT COURSEWORK-------
% \section{Relevant Coursework}
%  %\resumeSubHeadingListStart
%  %     \resumeSubHeadingListStart
%  %       \resumeProjectHeading
%  \textbf{{ Undergrad}}
% \vspace*{-0.5\multicolsep}
% \begin{multicols}{3}
% 	\begin{itemize}[itemsep=-5pt, parsep=3pt,leftmargin=0.2in]
% 		\item{ Intro to computer programming}
% 		\item{ Numerical analysis}
% 		\item{ Intro to computational physics}
% 		\item{ Wolfram programming language}
% 		\item{ Intro to electronics}
% 		\item{ Statistical physics}
% 		\item{ Intro to solid state physics}
% 		\item{ Quantum mechanics I, II}
% 		%\item{ Calculus I, II, III}
% 		%\item{ Analytic geometry}
% 		\item{ Linear algebra}
% 		\item{ Differential geometry}
% 		%\item{ Intro to mathematical physics}
% 		\item{ Mathematical physics I, II}
% 		\item{ Group theory and Lie algebras}
% 	\end{itemize}
% \end{multicols}
% \vspace*{-2\multicolsep}
% \vspaceDefaultAbove
% \textbf{{ Graduate}}
% \vspace*{-0.5\multicolsep}
% \begin{multicols}{3}
% 	\begin{itemize}[itemsep=-5pt, parsep=3pt,leftmargin=0.2in]
% 		\item{ Adv. statistical mechanics I, II}
% 		\item{ Adv. quantum mechanics}
% 		\item{ Adv. mathematical physics I}
% 		\item{ Quantum many-particle systems}
% 		\item{ Quantum field theory I, II}
% 		\item{ Solid state physics}
% 	\end{itemize}
% \end{multicols}
% \vspace*{-2\multicolsep}
% \vspaceDefaultAbove
% \textbf{{ Online}}
% \vspace*{-0.5\multicolsep}
% \begin{multicols}{2}
% 	\begin{itemize}[itemsep=-5pt, parsep=3pt,leftmargin=0.2in]
% 		\item{ Stanford CS229: machine learning, Andrew NG}
% 		\item{ NVIDIA DLI: fundamentals of deep learning} %[\href{https://courses.nvidia.com/certificates/f9656e22258b417e8d4fccbf8de26a76}{credential}] }
% 		\item{ Coursera: deep learning specialization, Andrew NG}
% 		\item{ MIT 9.520/6.860: statistical learning theory and applications}
% 	\end{itemize}
% \end{multicols}
% % \vspaceSectionAbove
% %         \vspace*{1.0\multicolsep}
% %\resumeSubHeadingListEnd
% %
% %\vspace{4pt}
% %
% % \today
\begin{tikzpicture}[remember picture,overlay]
    \node[anchor=south, yshift=0.2cm, text=gray] at (current page.south) {\small \textit{Updated \today}};
\end{tikzpicture}
% % 
% % {BJC HealthCare}{Feb. 2021}
% % {Data Manager}{Saint Louis, MO, USA}
% % \resumeItemListStart
% % 
% % \resumeItem{ Worked two days as data manager in a vaccination clinic from BJC HealthCare during the COVID-19 pandemic.}
\end{document}
